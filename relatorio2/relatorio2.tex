\documentclass[a4paper,12pt]{article}

\usepackage[utf8]{inputenc}
\usepackage[brazil]{babel}

\usepackage{url}
\usepackage{graphicx}

% Listings
\usepackage{listings}
\usepackage{color}

\definecolor{dkgreen}{rgb}{0,0.6,0}
\definecolor{gray}{rgb}{0.5,0.5,0.5}
\definecolor{mauve}{rgb}{0.58,0,0.82}

\renewcommand{\lstlistingname}{Anexo}
\lstset{frame=tb,
  language=C,
  aboveskip=3mm,
  belowskip=3mm,
  showstringspaces=false,
  columns=flexible,
  basicstyle={\small\ttfamily},
  numbers=none,
  numberstyle=\tiny\color{gray},
  keywordstyle=\color{blue},
  commentstyle=\color{dkgreen},
  stringstyle=\color{mauve},
  breaklines=true,
  breakatwhitespace=true,
  tabsize=3
}

% ASM packages
\usepackage{amsmath, amstext, amsfonts}

\setlength{\parskip}{0.9em}
\renewcommand{\baselinestretch}{1.2}

% CABEÇALHO
\title{Desenvolvimento de uma política de escalonamento de loteria para o kernel Linux}
\author{
    Arthur Xavier\\
    \small{xavier@dcc.ufmg.br}
    \and
    Rodrigo Gontijo\\
    \small{rodrigo.gontijo@dcc.ufmg.br}
    \and
    Caio Godoy\\
    \small{caio.godoy@dcc.ufmg.br}
}

\begin{document}

\maketitle


%%%%%%%%%%%%%%%%%%%%%%%%%%%%%%%%%%%%%%%%%%%%%%%%%%%%%%%%%%%%%%%%%%%%%%%%%%%%%%%
% INTRODUÇÃO
% Retomada do problema e da solução proposta
%%%%%%%%%%%%%%%%%%%%%%%%%%%%%%%%%%%%%%%%%%%%%%%%%%%%%%%%%%%%%%%%%%%%%%%%%%%%%%%
\section{Introdução}
O escalonador do Linux é baseado em compartilhamento de tempo por prioridades, ou seja cada tarefa é executada até que a sua fatia de tempo, também chamada de quantum, expire, e um processo de prioridade mais alta torne-se executável ou o processo atual bloqueie.

O kernel Linux utiliza três diferentes políticas de escalonamento, cada qual atende a tipos diferentes de processo: o \emph{Escalonador First In First Out}, válido apenas para processos de tempo real; o \emph{Escalonador Round Robin}, comumente utilizado em sistemas de tempo compartilhado e implementado utilizando-se uma fila circular; e o \emph{Escalonador Other}, o qual corresponde a uma fila com vários níveis de prioridades dinâmicas com tempo compartilhado para cada processo, utilizado para processos interativos e \emph{batch}.

O problema, entretanto, com as políticas utilizadas pelo escalonador de processos do kernel Linux é a possibilidade de ocorrência de \emph{starvation} em ambientes multi-processados.

A solução proposta neste trabalho foi baseada na política de escalonamento de loteria\cite{LSWikipedia2015}, pois o problema de \emph{starvation} pode serresolvido com a distribuição de prioridades não-nulas. Uma vantagem da política de escalonamento de loteria é que, distribuindo-se mais bilhetes para um determinado processo, aumenta-se as chances de este ser selecionado, sendo trivial implementar seleção por prioridades (não determinística).


%%%%%%%%%%%%%%%%%%%%%%%%%%%%%%%%%%%%%%%%%%%%%%%%%%%%%%%%%%%%%%%%%%%%%%%%%%%%%%%
% AMBIENTE
% Sobre o ambiente de desenvolvimento utilizado
%%%%%%%%%%%%%%%%%%%%%%%%%%%%%%%%%%%%%%%%%%%%%%%%%%%%%%%%%%%%%%%%%%%%%%%%%%%%%%%
\section{Implementação}

\subsection{Ambiente}
Como primeiro passo para o começo do desenvolvimento de nossa proposta, acordamos na escolha de uma versão específica do kernel Linux. A versão Linux 3.2 \cite{Kernel2016} foi escolhida por ter boa documentação. Para o desenvolvimento e testes do sistema operacional, escolhemos utilizar o \emph{Ubuntu Linux 12.04 LTS} instalado em uma máquina virtual no \emph{Virtual Box}.

\subsection{Controle e distribuição de bilhetes}
Inicialmente, pela inexperiência e desconhecimento do extenso código do kernel Linux, optamos por implementar a estrutura de controle dos bilhetes ao invés de realizar quaisquer alterações no código base do kernel, enquanto nos adaptamos e conhecemos mais o código.

Houveram muitas discussões sobre a maneira como os bilhetes e sua distribuição deveriam ser implementados, pois a escolha da estrutura de dados influenciaria diretamente no desempenho do novo escalonador. Um problema inicial é a alocação de bilhetes para um dado processo. Uma solução ingênua seria a criação de uma lista contendo os índices de todos os bilhetes de um processo. Um problema desta implementação é o alto custo de pesquisa, dado que uma vez sorteado um bilhete, no pior caso, deve-se pesquisar dentre todos os bilhetes que já foram distribuídos (para todos os processos).

A estrutura de dados proposta e implementada foi, então, uma distribuição de bilhetes de forma que um processo somente possui bilhetes numerados em um intervalo de $[x, y],~ x, y \in \mathbb{N}$. O conjunto $B_n$ de bilhetes de um dado processo $n$ é então da forma $B_n = \{ i ~|~ i \in [x_n, y_n] \}$.

\begin{lstlisting}[caption=Estrutura de distribuição de bilhetes]
struct t_tickets {
  long start;
  long end;
};
\end{lstlisting}

Embora o problema da consulta seja resolvido com esta estrutura, outros surgem, por exemplo: o que deve ser feito com os bilhetes de um processo terminado? Como realocar os bilhetes de um processo que teve sua prioridade alterada? Como lidar com o caso de nenhum processo ser sorteado devido a buracos na distribuição de bilhetes?

Estes não são problemas simples de serem resolvidos, pois qualquer estrutura de dados implica em um \emph{tradeoff} em algum ponto. Felizmente a maturidade do kernel Linux foi de grande ajuda neste momento, pois os dois primeiros problemas foram resolvidos com a utilização de uma estrutura já presente no kernel e extensivamente utilizada: o \emph{slab allocator}, uma lista que é utilizada como uma cache para alocação dinâmica.

\subsection{Distribuição de prioridades}
A distribuição e o conceito de prioridades de execução para processos, no contexto da política de loteria, é natural e inata da política. A prioridade de um processo é diretamente proporcional ao número de bilhetes que possui. Desta forma, segundo a estrutura de distribuição de bilhetes utilizadas, seria correto utilizar uma estrutura semelhante para a distribuição de prioridades, isto é, baseada em intervalos.

No kernel Linux, a prioridade de um processo varia de 1 a 10, sendo 1 o mais prioritário e 10 o menos prioritário. Desta forma foi estabelecido que os bilhetes serão distribuídos em múltiplos de 10, tal que um processo de prioridade 1 possui 100 bilhetes enquanto um processo de prioridade 10 possui apenas 10 bilhetes.


%%%%%%%%%%%%%%%%%%%%%%%%%%%%%%%%%%%%%%%%%%%%%%%%%%%%%%%%%%%%%%%%%%%%%%%%%%%%%%%
% PRÓXIMOS OBJETIVOS
%%%%%%%%%%%%%%%%%%%%%%%%%%%%%%%%%%%%%%%%%%%%%%%%%%%%%%%%%%%%%%%%%%%%%%%%%%%%%%%
\section{Próximos objetivos}
Na próxima etapa do trabalho, com o ambiente estabelecido e o código do kernel bem estudado, devemos dar continuidade à implementação das funcionalidades restantes para o correto funcionamento do escalonador. Para tal, resta-nos, segundo o planejamento inicial:

\begin{enumerate}
  \item Implementar o controle dos bilhetes, isto é, as funções de distribuição e gerenciamento dos bilhetes baseadas na estrutura já implementada. Planejamos utilizar uma estrutura de \emph{buffer} baseada em listas encadeadas para a geração, distribuição e consulta de bilhetes;
  \item Implementação do escalonador em si, através da instanciação de uma estrutura \emph{sched\_class}, a qual deve conter a nova política de escalonamento, funções de gerenciamento da fila de processos e verificação de preempção de processos. Deverá ser implementada também, uma nova estrutura para a fila de processos, que funcionará integrada à nova política implementada. O escalonador será o responsável por escolher aleatóriamente um novo processo a ser executado;
  \item Alterar as funções e estruturas já existentes, isto é, conectar o novo escalonador ao kernel.
\end{enumerate}


%%%%%%%%%%%%%%%%%%%%%%%%%%%%%%%%%%%%%%%%%%%%%%%%%%%%%%%%%%%%%%%%%%%%%%%%%%%%%%%
% CONCLUSÃO
% Conclusão/resumo geral e resultados esperados
%%%%%%%%%%%%%%%%%%%%%%%%%%%%%%%%%%%%%%%%%%%%%%%%%%%%%%%%%%%%%%%%%%%%%%%%%%%%%%%
\section{Conclusão}
O tamanho do código do kernel Linux foi suficiente para que grande parte do tempo inicial do desenvolvimento do trabalho fosse gasto com o estudo e entendimento do código. A configuração e utilização do ambiente (máquina virtual e sistema operacional) também se mostraram empecilhos para a implementação e testes.

Às hipóteses e teorias desenvolvidas nesta etapa resta, portanto, uma validação. Espera-se poder concluir a implementação do escalonador com a nova política e do gerenciamento e distribuição de bilhetes e prioridades nesta próxima etapa do trabalho.


%%%%%%%%%%%%%%%%%%%%%%%%%%%%%%%%%%%%%%%%%%%%%%%%%%%%%%%%%%%%%%%%%%%%%%%%%%%%%%%
% REFERÊNCIAS
%%%%%%%%%%%%%%%%%%%%%%%%%%%%%%%%%%%%%%%%%%%%%%%%%%%%%%%%%%%%%%%%%%%%%%%%%%%%%%%
\begin{thebibliography}{9}
\bibitem{LSWikipedia2015} Wikipedia. \emph{Lottery Scheduling}. 2015. \url{https://en.wikipedia.org/wiki/Lottery_scheduling}
\bibitem{Kernel2016} Linux Kernel Organization. \emph{The Linux Kernel Archives}. 2016. \url{https://www.kernel.org/}
\end{thebibliography}

\end{document}
