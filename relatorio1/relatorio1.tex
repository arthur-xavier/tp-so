\documentclass[a4paper,12pt]{article}

\usepackage[utf8]{inputenc}
\usepackage[brazil]{babel}

\usepackage{url}
\usepackage{graphicx}

% ASM packages
\usepackage{amsmath, amstext}

% CABEÇALHO
\title{Proposta de uma política de escalonamento de loteria para o kernel Linux}
\author{
    Arthur Xavier\\
    \small{xavier@dcc.ufmg.br}
    \and
    Rodrigo Gontijo\\
    \small{@dcc.ufmg.br}
    \and
    Caio Godoy\\
    \small{@dcc.ufmg.br}
}

\begin{document}

\maketitle

%%%%%%%%%%%%%%%%%%%%%%%%%%%%%%%%%%%%%%%%%%%%%%%%%%%%%%%%%%%%%%%%%%%%%%%%%%%%%%%
% RESUMO
%%%%%%%%%%%%%%%%%%%%%%%%%%%%%%%%%%%%%%%%%%%%%%%%%%%%%%%%%%%%%%%%%%%%%%%%%%%%%%%
\begin{abstract}
  Este trabalho tem como objetivo propor uma melhoria funcional para o kernel do sistema operacional Linux. Escolhemos propor uma mudança no escalonador de processos do sistema operacional, mais especificamente, uma troca da política de escalonamento utilizada pelo respectivo escalonador.
\end{abstract}

%%%%%%%%%%%%%%%%%%%%%%%%%%%%%%%%%%%%%%%%%%%%%%%%%%%%%%%%%%%%%%%%%%%%%%%%%%%%%%%
% INTRODUÇÃO
% O que é o linux?
% Pequena descrição e vantagens
%%%%%%%%%%%%%%%%%%%%%%%%%%%%%%%%%%%%%%%%%%%%%%%%%%%%%%%%%%%%%%%%%%%%%%%%%%%%%%%
\section{Introdução}

%%%%%%%%%%%%%%%%%%%%%%%%%%%%%%%%%%%%%%%%%%%%%%%%%%%%%%%%%%%%%%%%%%%%%%%%%%%%%%%
% PROBLEMA
% Abordagem do tema/problema
%%%%%%%%%%%%%%%%%%%%%%%%%%%%%%%%%%%%%%%%%%%%%%%%%%%%%%%%%%%%%%%%%%%%%%%%%%%%%%%
\section{Problema}

%%%%%%%%%%%%%%%%%%%%%%%%%%%%%%%%%%%%%%%%%%%%%%%%%%%%%%%%%%%%%%%%%%%%%%%%%%%%%%%
% SOLUÇÃO
% Solução abordada
%%%%%%%%%%%%%%%%%%%%%%%%%%%%%%%%%%%%%%%%%%%%%%%%%%%%%%%%%%%%%%%%%%%%%%%%%%%%%%%
\section{Solução}

%%%%%%%%%%%%%%%%%%%%%%%%%%%%%%%%%%%%%%%%%%%%%%%%%%%%%%%%%%%%%%%%%%%%%%%%%%%%%%%
% CONCLUSÃO
% Conclusão/resumo geral e resultados esperados
%%%%%%%%%%%%%%%%%%%%%%%%%%%%%%%%%%%%%%%%%%%%%%%%%%%%%%%%%%%%%%%%%%%%%%%%%%%%%%%
\section{Conclusão}

%%%%%%%%%%%%%%%%%%%%%%%%%%%%%%%%%%%%%%%%%%%%%%%%%%%%%%%%%%%%%%%%%%%%%%%%%%%%%%%
% REFERÊNCIAS
%%%%%%%%%%%%%%%%%%%%%%%%%%%%%%%%%%%%%%%%%%%%%%%%%%%%%%%%%%%%%%%%%%%%%%%%%%%%%%%
\begin{thebibliography}{99}
\bibitem{LinuxWikipedia} Wikipedia. \emph{Linux}. 2016. \url{https://en.wikipedia.org/wiki/Linux}
\end{thebibliography}

\end{document}
