\documentclass[a4paper,12pt]{article}

\usepackage[utf8]{inputenc}
\usepackage[brazil]{babel}

\usepackage{url}
\usepackage{graphicx}

% ASM packages
\usepackage{amsmath, amstext}

\setlength{\parskip}{0.9em}
\renewcommand{\baselinestretch}{1.2}

% CABEÇALHO
\title{Proposta de uma política de escalonamento de loteria para o kernel Linux}
\author{
    Arthur Xavier\\
    \small{xavier@dcc.ufmg.br}
    \and
    Rodrigo Gontijo\\
    \small{@dcc.ufmg.br}
    \and
    Caio Godoy\\
    \small{@dcc.ufmg.br}
}

\begin{document}

\maketitle

%%%%%%%%%%%%%%%%%%%%%%%%%%%%%%%%%%%%%%%%%%%%%%%%%%%%%%%%%%%%%%%%%%%%%%%%%%%%%%%
% RESUMO
%%%%%%%%%%%%%%%%%%%%%%%%%%%%%%%%%%%%%%%%%%%%%%%%%%%%%%%%%%%%%%%%%%%%%%%%%%%%%%%
\begin{abstract}
  Este trabalho tem como objetivo propor uma melhoria funcional para o kernel do sistema operacional Linux. Escolhemos propor uma mudança no escalonador de processos do sistema operacional, mais especificamente, uma troca da política de escalonamento utilizada pelo respectivo escalonador.
\end{abstract}

%%%%%%%%%%%%%%%%%%%%%%%%%%%%%%%%%%%%%%%%%%%%%%%%%%%%%%%%%%%%%%%%%%%%%%%%%%%%%%%
% INTRODUÇÃO
% O que é o linux?
% Pequena descrição e vantagens
%%%%%%%%%%%%%%%%%%%%%%%%%%%%%%%%%%%%%%%%%%%%%%%%%%%%%%%%%%%%%%%%%%%%%%%%%%%%%%%
\section{Introdução}
Linux é o software Open Source mais conhecido e mais utilizado do mundo. Se desenvolveu em um sistema operacional para negócios, educação e de uso pessoal. É um sistema operacional de propósito geral de código aberto. \cite{Welsh1998}

A primeira versão do kernel Linux foi publicada em 5 de outubro de 1991 por Linus Torvalds como um clone do sistema operacional UNIX. Linux foi desenvolvido originalmente como um sistema operacional gratuito para computadores pessoais baseado na arquitetura Intel x86, mas desde então tem sido portado para mais plataformas do que qualquer outro sistema operacional. \cite{LinuxWikipedia2016}

A utilização de um sistema operacional baseado em Linux (chamadas \emph{distribuições} ou \emph{distros}) provê diversas vantagens para os usuários e também para os desenvolvedores e mantenedores destes sistemas, dentre elas, principalmente, liberdade, estabilidade e segurança.

Por ser um sistema operacional de código aberto, Linux é gratuito, e possui diversas versões e distribuições. O usuário pode escolher qual das milhares de distribuições pode atender melhor a seu uso específico e pode, inclusive, escrever sua própria versão ou alterar uma já existente. Estabilidade e segurança também são fatores que decorre de sua origem Open Source. Por ter milhares de mantenedores e desenvolvedores ativos nos projetos Linux, o sistema operacional possui um forte atestado de funcionamento, o que garante sua estabilidade. Devido a este fato também, sistemas Linux possuem repositórios oficiais de software, o que garante que estes softwares não são nocivos ao sistema ou à plataforma.

%%%%%%%%%%%%%%%%%%%%%%%%%%%%%%%%%%%%%%%%%%%%%%%%%%%%%%%%%%%%%%%%%%%%%%%%%%%%%%%
% PROBLEMA
% Abordagem do tema/problema
%%%%%%%%%%%%%%%%%%%%%%%%%%%%%%%%%%%%%%%%%%%%%%%%%%%%%%%%%%%%%%%%%%%%%%%%%%%%%%%
\section{Problema}
O escalonador do Linux é baseado em compartilhamento de tempo por prioridades, ou seja cada tarefa é executada até que a sua fatia de tempo, também chamada de quantum, expire, e um processo de prioridade mais alta torne-se executável ou o processo atual bloqueie. O kernel Linux utiliza três diferentes políticas de escalonamento, cada qual atende a tipos diferentes de processo.

\subsection{Escalonador FIFO}
A primeira política de escalonamento utilizada pelo Linux é a do escalonador FIFO (First In First Out). Essa política é válida apenas para processos de tempo real e prevê que quando um processo é alocado ao processador, ele executa normalmente até que uma de quatro situações ocorra: um processo também de tempo real e de prioridade mais alta for apto a ser executado, assim o processo cede o processador para essa nova tarefa, ou o processo libera espontaneamente o processador para um processo de prioridade igual à sua, ou o processador é bloqueado em uma operação de entrada e saída e na última hipótese, o processo termina. \cite{Oliveira2008}

\subsection{Escalonador Round Robin}
Outra política existente é a do escalonador Round Robin (RR) comumente utilizado em sistemas de tempo compartilhado. O sistema operacional determina uma quantidade de tempo o quantum, que diz o tempo que o processador terá para trabalhar com o processo e é criada uma fila circular onde os processos serão incluídos. Com isso cada tarefa da fila é trabalhada até que o seu quantum expire e o sistema pare-as, sofrendo assim a preempção, e logo em seguida a próxima toma o lugar, fazendo isso até que todas sejam finalizadas. Essa política também é usada apenas em processos de tempo real. \cite{Deitel2006}

\subsection{Escalonador Other}
A última política utilizada pelo kernel Linux corresponde a uma fila com vários níveis de prioridades dinâmicas com tempo compartilhado para cada processo. Os processos interativos e \emph{batch} (processos estáticos) são escalonados através deste escalonador (\emph{sched\_other}).

\subsection{Starvation}
O problema, entretanto, com as políticas utilizadas pelo escalonador de processos do kernel Linux é a possibilidade de ocorrência de \emph{starvation} em ambientes multi-processados, problema este que pode ser resolvido com a distribuição de prioridades não-nulas, como implementado na política de escalonamento de loteria\cite{LSWikipedia2015}.

%%%%%%%%%%%%%%%%%%%%%%%%%%%%%%%%%%%%%%%%%%%%%%%%%%%%%%%%%%%%%%%%%%%%%%%%%%%%%%%
% SOLUÇÃO
% Solução abordada
%%%%%%%%%%%%%%%%%%%%%%%%%%%%%%%%%%%%%%%%%%%%%%%%%%%%%%%%%%%%%%%%%%%%%%%%%%%%%%%
\section{Solução}
Para solucionar o problema da inanição nas políticas de escalonamento, escolhemos implementar a política escalonamento de loteria. Esta política funciona da seguinte maneira:

\begin{enumerate}
  \item Bilhetes são distribuídos aleatoriamente (ou segundo alguma distribuição probabilística) a processos;
  \item Um sorteio de bilhetes é feito aleatoriamente;
  \item Ao processo cujo bilhete foi escolhido, é, em seguida, dado uma fatia de tempo de excução;
  \item O quantum pode variar devido ao número de sorteios por segundo.
\end{enumerate}

Uma vantagem da política de escalonamento de loteria é que, distribuindo-se mais bilhetes para um determinado processo, aumenta-se as chances de este ser selecionado, sendo trivial implementar seleção por prioridades (não determinística). Uma outra vantagem dessa política é que ela elimina o problema da inanição, já que todos os processos têm a chance de ser executados, mesmo podendo esta ser mínima.

Para implementar essa política de escalonamento, devemos seguir os seguintes passos:

\begin{enumerate}
  \item Alterar as funções e estruturas já existentes e realizar as adições necessárias, assim o escalonador irá suportar essa política;
  \item Implementar o controle dos bilhetes, sendo que esse controle pode ser um array contendo as informações necessárias para se construir um bilhete;
  \item Implementação do escalonador em si, podendo este ser feito como uma lista de processos e suas funções de manipulação.
\end{enumerate}

%%%%%%%%%%%%%%%%%%%%%%%%%%%%%%%%%%%%%%%%%%%%%%%%%%%%%%%%%%%%%%%%%%%%%%%%%%%%%%%
% CONCLUSÃO
% Conclusão/resumo geral e resultados esperados
%%%%%%%%%%%%%%%%%%%%%%%%%%%%%%%%%%%%%%%%%%%%%%%%%%%%%%%%%%%%%%%%%%%%%%%%%%%%%%%
\section{Conclusão}
Com a implementação de uma nova política de escalonamento baseada no método \emph{Lottery Scheduling}, espera-se uma melhora significativa no tempo de processamento em ambientes multi-processados Linux (como computadores pessoais), devido à diminuição da quantidade de ocorrências de inanição (\emph{starvation}). Esta diminuição ocasionaria, obviamente, uma redução do tempo de processamento devido à diminuição de tentativas de processamento por processos que sofreram de inanição em algum intervalo de tempo.

%%%%%%%%%%%%%%%%%%%%%%%%%%%%%%%%%%%%%%%%%%%%%%%%%%%%%%%%%%%%%%%%%%%%%%%%%%%%%%%
% REFERÊNCIAS
%%%%%%%%%%%%%%%%%%%%%%%%%%%%%%%%%%%%%%%%%%%%%%%%%%%%%%%%%%%%%%%%%%%%%%%%%%%%%%%
\begin{thebibliography}{9}
\bibitem{Welsh1998} Matt Welsh et al. \emph{Linux Installation and Getting Started}. 1998. \url{http://www.tldp.org/LDP/gs/gs.html}
\bibitem{LinuxWikipedia2016} Wikipedia. \emph{Linux}. 2016. \url{https://pt.wikipedia.org/wiki/Linux}
\bibitem{Oliveira2008} Rômulo S. de Oliveira, Alexandre da S. Carissimi, Simão S. Toscani. \emph{Sistemas Operacionais}. 2008.
\bibitem{Deitel2006} Harvey M. Deitel, Paul J. Deitel, David R. Choffnes. \emph{Operating Systems}. 2006.
\bibitem{LSWikipedia2015} Wikipedia. \emph{Lottery Scheduling}. 2015. \url{https://en.wikipedia.org/wiki/Lottery_scheduling}
\end{thebibliography}

\end{document}
