\documentclass[a4paper,12pt]{article}

\usepackage[utf8]{inputenc}
\usepackage[brazil]{babel}

\usepackage{url}
\usepackage{graphicx}

% ASM packages
\usepackage{amsmath, amstext}

% CABEÇALHO
\title{Proposta de uma política de escalonamento de loteria para o kernel Linux}
\author{
    Arthur Xavier\\
    \small{xavier@dcc.ufmg.br}
    \and
    Rodrigo Gontijo\\
    \small{@dcc.ufmg.br}
    \and
    Caio Godoy\\
    \small{@dcc.ufmg.br}
}

\begin{document}

\maketitle

%%%%%%%%%%%%%%%%%%%%%%%%%%%%%%%%%%%%%%%%%%%%%%%%%%%%%%%%%%%%%%%%%%%%%%%%%%%%%%%
% RESUMO
%%%%%%%%%%%%%%%%%%%%%%%%%%%%%%%%%%%%%%%%%%%%%%%%%%%%%%%%%%%%%%%%%%%%%%%%%%%%%%%
\begin{abstract}
  Este trabalho tem como objetivo propor uma melhoria funcional para o kernel do sistema operacional Linux. Escolhemos propor uma mudança no escalonador de processos do sistema operacional, mais especificamente, uma troca da política de escalonamento utilizada pelo respectivo escalonador.
\end{abstract}

%%%%%%%%%%%%%%%%%%%%%%%%%%%%%%%%%%%%%%%%%%%%%%%%%%%%%%%%%%%%%%%%%%%%%%%%%%%%%%%
% INTRODUÇÃO
% O que é o linux?
% Pequena descrição e vantagens
%%%%%%%%%%%%%%%%%%%%%%%%%%%%%%%%%%%%%%%%%%%%%%%%%%%%%%%%%%%%%%%%%%%%%%%%%%%%%%%
\section{Introdução}

\pagebreak

%%%%%%%%%%%%%%%%%%%%%%%%%%%%%%%%%%%%%%%%%%%%%%%%%%%%%%%%%%%%%%%%%%%%%%%%%%%%%%%
% PROBLEMA
% Abordagem do tema/problema
%%%%%%%%%%%%%%%%%%%%%%%%%%%%%%%%%%%%%%%%%%%%%%%%%%%%%%%%%%%%%%%%%%%%%%%%%%%%%%%
\section{Problema}
O escalonador do Linux é baseado em compartilhamento de tempo por prioridades, ou seja cada tarefa é executada até que a sua fatia de tempo, também chamada de quantum, expire, e um processo de prioridade mais alta torne-se executável ou o processo atual bloqueie. O kernel Linux utiliza três diferentes políticas de escalonamento, cada qual atende a tipos diferentes de processo.

\subsection{Escalonador FIFO}
A primeira política de escalonamento utilizada pelo Linux é a do escalonador FIFO (First In First Out). Essa política é válida apenas para processos de tempo real e prevê que quando um processo é alocado ao processador, ele executa normalmente até que uma de quatro situações ocorra: um processo também de tempo real e de prioridade mais alta for apto a ser executado, assim o processo cede o processador para essa nova tarefa, ou o processo libera espontaneamente o processador para um processo de prioridade igual à sua, ou o processador é bloqueado em uma operação de entrada e saída e na última hipótese, o processo termina. \cite{Oliveira2008}

\subsection{Escalonador Round Robin}
Outra política existente é a do escalonador Round Robin (RR) comumente utilizado em sistemas de tempo compartilhado. O sistema operacional determina uma quantidade de tempo o quantum, que diz o tempo que o processador terá para trabalhar com o processo e é criada uma fila circular onde os processos serão incluídos. Com isso cada tarefa da fila é trabalhada até que o seu quantum expire e o sistema pare-as, sofrendo assim a preempção, e logo em seguida a próxima toma o lugar, fazendo isso até que todas sejam finalizadas. Essa política também é usada apenas em processos de tempo real. \cite{Deitel2006}

\subsection{Escalonador Other}
A última política utilizada pelo kernel Linux corresponde a uma fila com vários níveis de prioridades dinâmicas com tempo compartilhado para cada processo. Os processos interativos e \emph{batch} (processos estáticos) são escalonados através deste escalonador (\emph{sched\_other}).

\pagebreak

O problema, entretanto, com as políticas utilizadas pelo escalonador de processos do kernel Linux é a possibilidade de ocorrência de \emph{starvation}, problema este que pode ser resolvido com a distribuição de prioridades não-nulas, como implementado na política de escalonamento de loteria (\emph{lottery scheduling}).

%%%%%%%%%%%%%%%%%%%%%%%%%%%%%%%%%%%%%%%%%%%%%%%%%%%%%%%%%%%%%%%%%%%%%%%%%%%%%%%
% SOLUÇÃO
% Solução abordada
%%%%%%%%%%%%%%%%%%%%%%%%%%%%%%%%%%%%%%%%%%%%%%%%%%%%%%%%%%%%%%%%%%%%%%%%%%%%%%%
\section{Solução}

%%%%%%%%%%%%%%%%%%%%%%%%%%%%%%%%%%%%%%%%%%%%%%%%%%%%%%%%%%%%%%%%%%%%%%%%%%%%%%%
% CONCLUSÃO
% Conclusão/resumo geral e resultados esperados
%%%%%%%%%%%%%%%%%%%%%%%%%%%%%%%%%%%%%%%%%%%%%%%%%%%%%%%%%%%%%%%%%%%%%%%%%%%%%%%
\section{Conclusão}

%%%%%%%%%%%%%%%%%%%%%%%%%%%%%%%%%%%%%%%%%%%%%%%%%%%%%%%%%%%%%%%%%%%%%%%%%%%%%%%
% REFERÊNCIAS
%%%%%%%%%%%%%%%%%%%%%%%%%%%%%%%%%%%%%%%%%%%%%%%%%%%%%%%%%%%%%%%%%%%%%%%%%%%%%%%
\begin{thebibliography}{9}
\bibitem{LinuxWikipedia2016} Wikipedia. \emph{Linux}. 2016. \url{https://en.wikipedia.org/wiki/Linux}
\bibitem{Oliveira2008} Rômulo S. de Oliveira, Alexandre da S. Carissimi, Simão S. Toscani. \emph{Sistemas Operacionais}. 2008.
\bibitem{Deitel2006} Harvey M. Deitel, Paul J. Deitel, David R. Choffnes. \emph{Operating Systems}. 2006.
\end{thebibliography}

\end{document}
